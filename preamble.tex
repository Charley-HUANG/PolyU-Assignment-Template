%---常用包导入,依托毕业论文导言区---
\usepackage{fullpage}
\usepackage{xcolor} %文字颜色
\usepackage{fontspec}

\usepackage{setspace}
% \doublespace % 设置行距 https://texblog.org/2011/09/30/quick-note-on-line-spacing/
% \linespread{1.5}

\usepackage{graphicx} % 图片
\graphicspath{{figures/}}

\usepackage{booktabs} % 表格
\usepackage{multirow} %EXCEL导出latex表格,需要这个包
\usepackage{caption}
\usepackage{array} %为了居中
\newcolumntype{C}[1]{>{\centering\arraybackslash}m{#1}}
\usepackage{threeparttable} %表格尾注

\usepackage{amssymb} %为了surd
\usepackage{amsmath} % displaymath出问题就导这个包
\usepackage{amsthm}
\newtheorem{theorem}{Theorem}[section]
% \newtheorem{lemma}{Lemma}[theorem]
\newtheorem{definition}[theorem]{Definition}
\newtheorem*{remark}{Remark}

\usepackage{hyperref}
\hypersetup{
     colorlinks   = true,
	linkcolor    = blue,
     citecolor    = blue
}

\usepackage[counterclockwise]{rotating}

\usepackage{siunitx} % 按数字位数对齐

\makeatletter
\newcommand\specialsectioning{\setcounter{secnumdepth}{-2}}
\makeatother

\usepackage[super,square,sort]{natbib}
\usepackage{lipsum}

\newcommand{\cmt}[1]{} % 注释

\setlength{\parindent}{0pt} % 全局首行不缩进

% ---自建问题,证明,解答环境---
\newenvironment{prob}[1]{\vskip .25cm {\bf Problem #1}}{\vskip .5cm}
\newenvironment{prof}[1]{\vskip .25cm \textit {Proof:} #1}{\begin{flushright}{\qedsymbol}\end{flushright}}
\newenvironment{solu}[1]{\vskip .25cm \textit {Solution:} #1}{}

%Page setup
\usepackage{lastpage}
\usepackage{fancyhdr}
\pagestyle{fancy}
\headheight 35pt
\lhead{\large \today}
\rhead{\includegraphics[width=5cm]{polyulogo.png}} % <-- school logo(please upload the file first, then change the name here)
\lfoot{}
\pagenumbering{arabic}
\cfoot{\small{\thepage \space of \space \pageref{LastPage}}}
\rfoot{}
\headsep 2em
\renewcommand{\baselinestretch}{1.25} 